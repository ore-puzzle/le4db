\documentclass{jarticle}
\usepackage[dvipdfmx]{graphicx}
\usepackage{here}
\usepackage{listings}

\begin{document}

\title{課題1}
\author{1029289895 尾崎翔太}
\date{2018/}

\maketitle
\newpage

\section{アプリケーションの説明}
ビデオレンタルシステムである. 一般ユーザはあるビデオはどの店にあるか, ある店にはどんなビデオがあるかなどを検索できる. 店側は一般ユーザがどのビデオを借りているか, 及びその返却期限を確認できる.
\section{利用者の役割}
\begin{description}
\item[一般ユーザ] \\ \leavevmode
このビデオレンタルショップを利用している消費者.
\item[店員] \\ \leavevmode
このビデオレンタルショップが有している店の従業員.
\item[管理者] \\ \leavevmode
このシステムの更新を行う人.
\end{description}
\section{役割ごとの機能}
\subsection{一般ユーザの機能}
\begin{description}
\item[ビデオ検索機能] \\ \leavevmode
ビデオの名前や店の名前やジャンルで検索して, ヒットしたビデオのリストを得ることができる. また, その結果を発売年や長さでソートできる.
\item[店検索機能]
ある特定のビデオについて, それが置いてある店を得ることができる.
\item[利用状況確認機能] \\ \leavevmode
自分がどのビデオを借りているか, 及びその返却期限を得ることができる.
\item[履歴確認機能] \\ \leavevmode
自分の今までの利用履歴を確認できる.
\end{description}
\subsection{店員の機能}
\begin{description}
\item[利用状況確認機能] \\ \leavevmode
自分が勤めている店について, どのビデオが誰に借りられていて, その返却期限はいつかを得ることができる.
\item[催促機能] \\ \leavevmode
返却期限を過ぎている一般ユーザに返却を促すメールを送ることができる.
\end{description}
\subsection{管理者}
\begin{description}
\item[ビデオ管理機能] \\ \leavevmode
新しいビデオを置いたり, 今まで置いてあったビデオを置かなくなったりしたときに追加/削除ができる.
\item[ユーザ管理機能] \\ \leavevmode
新しいユーザを登録できる.
\item[店管理機能] \\ \leavevmode
開店した店を追加したり, 閉店した店を削除したりできる.
\item[店員管理機能] \\ \leavevmode
新しい店員を登録したり, 辞めた店員を削除したりできる.
\end{description}
\section{実体関連図}

\end{document}
