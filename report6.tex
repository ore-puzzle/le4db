\documentclass{jarticle}
\usepackage[dvipdfmx]{graphicx}
\usepackage{here}
\usepackage{listings}

\begin{document}

\title{課題6}
\author{1029289895 尾崎翔太}
\date{2018/12/27}

\maketitle
\newpage

\section{キャッシュ}
\subsection{高速化}
id, name, priceからなる表testのデータを10000000件用意して
\begin{verbatim}
EXPLAIN ANALYSE SELECT id, name FROM test LIMIT 100;
\end{verbatim}
という文を繰り返し実行した.
結果はtable1.1のようになった.
\begin{table}[htbp]
  \begin{tabular}{|l||r|r|} \hline
    & Planning Time & Execution Time \\ \hline \hline
    1回目 & 268.038ms & 16.364ms \\ \hline
    2回目 & 0.189ms & 0.116ms \\ \hline
    3回目 & 0.167ms & 0.107ms \\ \hline
    4回目 & 0.121ms & 0.095ms \\ \hline
    5回目 & 0.150ms & 0.090ms \\ \hline
  \end{tabular}
  \caption{実行結果}
  \centering
\end{table}
1回目とそれ以降では明らかに差があることが見て取れる.
\subsection{キャッシュが効かないクエリ}
様々なクエリを試してみた結果, キーではない属性について選択するようなクエリはあまり早くならないことがわかった. これは, 選択は条件式のチェックが入るので, データがキャッシュされていても時間がかかるのかなと思った. キー属性が早くなるのは, 索引が作られていることと関係があるのかなと思った.
\section{索引の有無}
section1と同様のデータを10000行格納した表test10000, 100000行格納した表test100000, 1000000行格納した表test1000000を用いて
\begin{verbatim}
EXPLAIN ANALYSE SELECT * FROM (* 各表名 *) WHERE name = 'AGB';
\end{verbatim}
という文を, 索引の有無を変えて実行した. 結果はtable2.1のようになった.
\begin{table}[htbp]
  \begin{tabular}{|l||r|r|r|} \hline
  & test10000 & test100000 & test1000000 \\ \hline \hline
  索引なし Planning Time & 199.770ms & 108.441ms & 207.552ms \\ \hline
  索引なし Execution Time & 245.271ms & 2163.386ms & 13707.996ms \\ \hline
  索引あり Planning Time & 243.257ms & 162.324ms & 388.317ms \\ \hline
  索引あり Execution Time & 138.596ms & 357.031ms & 71.504ms \\ \hline
  \end{tabular}
  \centering
  \caption{実行結果(行数)}
\end{table}
次に, 属性数が2, 5, 10, 50である表test2, test5, test10, test50を用いて, section1と同じ文を実行した. 結果はtable2.1の通りである.
\begin{table}[htbp]
  \begin{tabular}{|l||r|r|r|r|} \hline
  & test2 & test5 & test10 & test50 \\ \hline \hline
  索引なし Planning Time & 224.698ms & 94.312ms & 145.610ms & 537.425ms \\ \hline
  索引なし Execution Time & 2165.536ms & 1258.916ms & 1620.546ms & 40167.935ms \\ \hline
  索引あり Planning Time & 216.681ms & 182.589ms & 210.753ms & 834.038ms \\ \hline
  索引あり Execution Time & 53.714ms & 67.667ms & 67. 776ms & 86.820ms \\ \hline
  \end{tabular}
  \centering
  \caption{実行結果(列数)}
\end{table}
行数の場合のように多い方がむしろ早くなるということはなかったが, どれくらい早くなるかという点においては, 列数が多い方が効果が見られる.
\section{選択率}
次の文によって生成された表を用いる.
\begin{verbatim}
CREATE TABLE test_rate (id INTEGER, data INTEGER, PRIMARY KEY (id));
\end{verbatim}
dataに0から999999までの整数が重複なくランダムに格納されたような1000000行のデータを用いる. この表に対して, 以下の文を実行する.
\begin{verbatim}
EXPLAIN ANALYSE SELECT * FROM test_rate WHERE data < 200000;
EXPLAIN ANALYSE SELECT * FROM test_rate WHERE data < 400000;
EXPLAIN ANALYSE SELECT * FROM test_rate WHERE data < 600000;
EXPLAIN ANALYSE SELECT * FROM test_rate WHERE data < 800000;
\end{verbatim}
上から順に, 選択率が0.2, 0.4, 0.6, 0.8の場合に対応している. 結果はtable3.1の通りである.
\begin{table}[htbp]
  \begin{tabular}{|l||r|r|r|r|} \hline
  & 0.2 & 0.4 & 0.6 & 0.8 \\ \hline \hline
  索引なし Planning Time & 92.786ms & 92.751ms & 170.530ms & 226.060ms \\ \hline
  索引なし Execution Time & 2305.741ms & 2217.160ms & 2350.566ms & 4772.773ms \\ \hline
  索引あり Planning Time & 678.754ms & 478.278ms & 989.707ms & 689.508ms \\ \hline
  索引あり Execution Time & 9386.464ms & 4876.238ms & 12009.779ms & 8265.277ms \\ \hline
  \end{tabular}
  \centering
  \caption{実行結果(選択率)}
\end{table}
索引を付けた方が時間がかかるというよくわからない結果になった. これまでと違って範囲選択であることが影響しているのかとも思ったが, B-treeは範囲選択にも対応しているはずなので, よくわからない. 選択率については, 索引なしでは0.8の場合だけ時間がかかっている一方, 索引ありでは時間はバラバラなので, 結局選択率による差はあまりないと思った.
\section{主索引と二次索引}
主索引は主キーに関する索引で, 二次索引はそれ以外の属性に関する索引である. それだけの差なので, 性能には違いがなさそうであるが, 主キーは必ずユニークであるので, その点で性能がよい. 二次索引もユニークであったり, 適切なインデックスを定めたりすることで性能がよくなることが期待される.

\section{参考文献}
\begin{itemize}
\item 「二次索引」, "https://www.kunihikokaneko.com/cc/db/7.html", 2018/12/27
\end{itemize}


\end{document}
