\documentclass{jarticle}
\usepackage[dvipdfmx]{graphicx}
\usepackage{here}
\usepackage{listings}

\begin{document}

\title{課題4}
\author{1029289895 尾崎翔太}
\date{2018/12/}

\maketitle
\newpage

\section{関係代数の射影および選択に対応するSQL文}
\subsection{SQL文}
\begin{verbatim}
SELECT title FROM content WHERE genre = 'movie';
\end{verbatim}
\subsection{説明}
表contentにおいて, genreがmovieであるもののtitleを得る.
\subsection{実行結果}
\begin{verbatim}
君の名は。
\end{verbatim}
\section{関係代数の自然結合に対応するSQL文}
\subsection{SQL文}
\begin{verbatim}
SELECT * FROM rent NATURAL JOIN duration;
\end{verbatim}
\subsection{説明}
表rentと表durationの自然結合を得る.
\subsection{実行結果}
\begin{verbatim}
yamada@example.jp|1|400|2018/4/10|2018/4/12|yes|2日
saitou@example.jp|2|700|2018/5/6|2018/5/13|yes|7日
takahashi@example.jp|3|380|2018/8/10|2018/8/11|yes|1日
takahashi@example.jp|4|380|2018/8/10|2018/8/11|yes|1日
isikawa@example.ne.jp|5|400|2018/12/18|2018/12/20|no|2日
\end{verbatim}
\section{UNIONを含むSQL文}
\subsection{SQL文}
\begin{verbatim}
SELECT * FROM work1 UNION ALL SELECT * FROM work2;
\end{verbatim}
\subsection{説明}
表work1と表work2の和集合を得る. なお, 和集合が取れる表がこれらのみだったので, この二つで和集合を取っただけで, 特に意味はない.
\subsection{実行結果}
\begin{verbatim}
1|B大学前店|B県g市き町2番地|walk
2|A交差点店|A県e市お町7番地|train
3|A大学前店|C県f市か町3番地|bicycle
4|A駅前店|A県a市あ町6番地|bicycle
5|B駅前店|B県c市く町1番地|bus
3|A大学前店|C県f市か町3番地|bus
1|B大学前店|B県g市き町2番地|近い
2|A交差点店|A県e市お町7番地|やりがいがある
3|A大学前店|C県f市か町3番地|温かい雰囲気
4|A駅前店|A県a市あ町6番地|やりがいがある
5|B駅前店|B県c市く町1番地|待遇が良い
3|A大学前店|C県f市か町3番地|待遇が良い
\end{verbatim}
\section{EXCEPTを含むSQL文}
\subsection{SQL文}
\begin{verbatim}
SELECT * FROM work1 EXCEPT SELECT * FROM work2;
\end{verbatim}
\subsection{説明}
表work1から表work2に現れるものを除いたものを得る. なお, UNIONの場合と同様の理由で, このSQL文に特に意味はない.
\subsection{実行結果}
\begin{verbatim}
1|B大学前店|B県g市き町2番地|walk
2|A交差点店|A県e市お町7番地|train
3|A大学前店|C県f市か町3番地|bicycle
3|A大学前店|C県f市か町3番地|bus
4|A駅前店|A県a市あ町6番地|bicycle
5|B駅前店|B県c市く町1番地|bus
\end{verbatim}
\section{DISTINCTを含むSQL文}
\subsection{SQL文}
\begin{verbatim}
SELECT DISTINCT mail FROM rent WHERE finished = 'no';
\end{verbatim}
\subsection{説明}
表rentにおいて, finishedがnoであるもののmailを重複なく得る.
\subsection{実行結果}
\begin{verbatim}
isikawa@example.ne.jp
\end{verbatim}
\section{集合関数(COUNT, SUM, AVG, MAX, MIN)を用いたSQL文}
\subsection{SQL文}
\begin{verbatim}
SELECT count(*) FROM rent WHERE mail = 'takahashi@example.jp';
\end{verbatim}
\subsection{説明}
表rentにおいて, mailがtakahashi@example.jpである行の数を得る.
\subsection{実行結果}
\begin{verbatim}
2
\end{verbatim}
\section{副質問を含むSQL文}
\subsection{SQL文}
\begin{verbatim}
\end{verbatim}
\subsection{説明}
\subsection{実行結果}
\begin{verbatim}
\end{verbatim}
\section{UPDATEを含むSQL文}
\subsection{SQL文}
\begin{verbatim}
\end{verbatim}
\subsection{説明}
\subsection{実行結果}
\begin{verbatim}
\end{verbatim}
\section{ORDER BYを含むSQL文}
\subsection{SQL文}
\begin{verbatim}
\end{verbatim}
\subsection{説明}
\subsection{実行結果}
\begin{verbatim}
\end{verbatim}
\section{CREATE VIEWを含むSQL文}
\subsection{SQL文}
\begin{verbatim}
\end{verbatim}
\subsection{説明}
\subsection{実行結果}
\begin{verbatim}
\end{verbatim}

\end{document}
