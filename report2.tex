\documentclass{jarticle}
\usepackage[dvipdfmx]{graphicx}
\usepackage{here}
\usepackage{listings}

\begin{document}

\title{課題2}
\author{1029289895 尾崎翔太}
\date{2018/}

\maketitle
\newpage

\section{関係スキーマ}
各関係スキーマは以下の通りである. 
\begin{itemize}
\item ユーザ(\underline{メールアドレス}, ユーザ名, ユーザ住所)
\item メディア(\underline{mid}, 種類)
\item 内容(\underline{題名}, \underline{発売年}, 長さ, 出版社, ジャンル)
\item 店(\underline{店名}, \underline{店住所}, 店長)
\item 店員(\underline{eid}, 店員名)
\item 借りている(\underline{メールアドレス}, \underline{題名}, \underline{発売年}, \underline{貸出日}, 料金)
\item 置いてある(\underline{題名}, \underline{発売年}, \underline{店名}, \underline{店住所}, 最大数, 数)
\item 働いている(\underline{eid}, \underline{店名}, \underline{店住所}, 勤務開始年)
\item 保存されている(\underline{mid}, \underline{題名}, \underline{発売年})
\end{itemize}
属性名は適宜変更した. また, 「働いている」に属性「勤務開始年」を追加した. ただし, 一度辞めて再び同じ店で働き出した場合には, 新しい方の情報を保持することとする.
\section{関数従属性/多値従属性}
\subsection{ユーザ}
\begin{description}
\item[\{メールアドレス\} $\rightarrow$ \{ユーザ名, ユーザ住所\}] \leavevmode \\
キー制約から自然に導かれる関数従属性である.
\end{description}
\subsection{メディア}
\begin{description}
\item[\{mid\} $\rightarrow$ \{種類\}] \leavevmode \\
キー制約から自然に導かれる関数従属性である.
\end{description}
\subsection{内容}
\begin{description}
\item[\{題名, 発売年\} $\rightarrow$ \{長さ, 出版社, ジャンル\}] \leavevmode \\
キー制約から自然に導かれる関数従属性である.
\end{description}
\subsection{店}
\begin{description}
\item[\{店名, 店住所\} $\rightarrow$ \{店長\}] \leavevmode \\
キー制約から自然に導かれる関数従属性である.
\end{description}
\subsection{店員}
\begin{description}
\item[\{eid\} $\rightarrow$ \{店員名\}] \leavevmode \\
キー制約から自然に導かれる関数従属性である.
\end{description}
\subsection{借りている}
\begin{description}
\item[\{メールアドレス, 題名, 発売年, 貸出日\} $\rightarrow$ \{料金\}] \leavevmode \\
キー制約から自然に導かれる関数従属性である.
\end{description}
\subsection{置いてある}
\begin{description}
\item[\{題名, 発売年, 店名, 店住所\} $\rightarrow$ \{最大数, 数\}] \leavevmode \\
キー制約から自然に導かれる関数従属性である.
\end{description}
\subsection{働いている}
\begin{description}
\item[\{eid, 店名, 店住所\} $\rightarrow$ \{勤務開始年\}] \leavevmode \\
キー制約から自然に導かれる関数従属性である.
\end{description}
\subsection{保存されている}
\begin{description}
\item[\{mid, 題名, 発売年\} $\rightarrow$ \{\}] \leavevmode \\
キー制約から自然に導かれる関数従属性である.
\end{description}
\end{document}



